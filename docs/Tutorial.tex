\documentclass[11pt]{article}


\usepackage{multirow}
\usepackage[T1]{fontenc}
\usepackage[nomarginpar,hmargin=2.5cm, vmargin=3cm]{geometry}
\usepackage[colorlinks=false]{hyperref}      
\usepackage{cleveref}     
\usepackage{xcolor}
\usepackage[document]{ragged2e}

\definecolor{listinggray}{gray}{0.9}

\definecolor{lightGray}{rgb}{.92,.92,.92}

\hypersetup{pdftex,colorlinks=true,allcolors=blue}
\renewcommand*\rmdefault{ppl}

\title{FGPU Tutorial \\ \large version 1.0}
\author{Muhammed Al Kadi}
\begin{document}
\maketitle

This tutorial helps new FGPU users to get used to its tool flow.
It offers a quick guide to configure, synthesize and program an FGPU core.
In addition, it describes how to install the required tools, e.g. the FGPU compiler and how to run the standard FGPU benchmarks in Xilinx SDK.
\section{Introduction}
Programming an application for FGPU goes through the following steps:
\begin{itemize}
  \item On the hardware side:
    \begin{itemize}
      \item The features of the FGPU core are to configured first in RTL (a \emph{.vhd} file), e.g. \# of CUs or 
        supported floating-point instructions.
      \item Then, a script implements FGPU with Vivado and generates:
      \begin{itemize}
        \item A hardware description file (\emph{.hdf}) that is needed to generate an SDK project.
        \item A bitstream to program the FPGA.
      \end{itemize}
    \end{itemize}
  \item On the software side:
    \begin{itemize}
      \item An OpenCL kernel is written to run on FGPU. 
        A script compiles it and generates some C files (\emph{.h} and \emph{.c}) to be included in an SDK project.
      \item A C/C++ host application is written to run on the ARM Cortex-A9.
      It should interface and control the FGPU processor.
    \end{itemize}
\end{itemize}

There are 3 versions of FGPU:
\begin{enumerate}
  \item \emph{V1} has two clock domains. It is less scalable than the others and has no thread-divergence support.
    The development of FGPU-V1 is discontinued.
  \item \emph{V2} works on a single clock domain. It is the most recent version.
  \item \emph{V3} is a branch of V2 and it supports partial configuration of floating-point logic.
\end{enumerate}
At the time of writing this tutorial, only V2 is added to the FGPU Github repository.

\section{Software Requirements}
Before you begin with this tutorial, 
make sure you have the following software installed on your computer:
\begin{itemize}
  \item \emph{Operating System:} you should have a Linux OS.
    The FGPU tool flow has been verified on Ubuntu 16.04.2 LTS 64bit and Fedora Workstation 22-25 64bit.
    It will definitely not work on Windows.
  \item \emph{Vivado Design Suite:} FGPU has been synthesized with the versions 2016.1-3 and 2017.2.
    Only Zynq-SoC device support is required and the Xilinx SDK should be installed as well.
  \item \emph{ModelSim or QuestaSim:}
    only needed for RTL simulations.
  \item \emph{A terminal emulator:} to communicate with the host application during execution, e.g. \emph{minicom}.
\end{itemize}

\begin{table}[t]
  \centering
  \caption{Required packages to setup an FGPU tool flow}
  \label{tab:packages}
  \begin{tabular}{c|c}
    OS    & Required Packages \\ \hline
    Ubuntu 16.04 - 64bit & ctags, cmake, g++, gawk, gcc-multilib, zlib1g-dev:i386\\ \hline
  \end{tabular}
  
\end{table}

\section{Tool Setup}
You may ignore this section if you are using the FGPU-VM.

\subsection{Download the FGPU Repository}
If you have \emph{git} installed, you may begin with checking out the FGPU repository from Github:

\centering \colorbox{lightGray}{ git clone https://github.com/malkadi/FGPU}
\justify
Otherwise, you can download and extract it manually to any local directory.
Any paths given in the rest of this tutorial are relative your installation directory.
\subsection{Install the Required OS Packages}
    For Ubuntu 16.04.2 64bit, you have to install the packages listed in \Cref{tab:packages}.
    This can be achieved by executing: \\
    \centering \colorbox{lightGray}{ sudo apt install \$(cat scripts/packages) }\\
\justify
Otherwise, you have to install the equivalent packages for your OS.

\subsection{Configure the Vivado Installation Path}
Please edit \colorbox{lightGray}{script/set\_paths.sh} to source the \colorbox{lightGray}{settings64.sh} script
which is included in your Vivado installation.

\subsection{Download and Compile LLVM}
First, the LLVM source code v3.7.1 with the clang compiler should be downloaded and extracted.
Then, the FGPU backend files in \colorbox{lightGray}{llvm-3.7.1.src.fgpu} can be copied to the LLVM source directory.
Finally, the whole LLVM code can be compiled to generate the compiler executables.
All these steps are automated in \colorbox{lightGray}{scripts/download\_and\_compile\_llvm.sh}.

\subsection{Compile FGPU Benchmarks for Xilinx SDK}
Before running any application located in the directory \colorbox{lightGray}{benchmark}, you need to:
\begin{enumerate}
  \item Create the HW and BSP projects by executing \colorbox{lightGray}{scripts/create\_sdk\_bsps.sh}.
    These projects will be created in hidden directories, e.g. \emph{.FGPU\_V2\_bsp}.
  \item Create and compile all FGPU benchmarks by executing \colorbox{lightGray}{scripts/create\_sdk\_project.sh}
\end{enumerate}
Moreover, the \colorbox{lightGray}{benchmark} folder includes two non-FGPU-relevant folders:
\begin{enumerate}
  \item \colorbox{lightGray}{benchmark/MicroBlaze} includes an SDK project that can be used to execute any of the FGPU benchmarks 
    on a MicroBlaze processor.
    The MicroBlaze runs on 180MHz and it has been optimized for best performance.
  \item \colorbox{lightGray}{benchmark/HLS} includes some of the FGPU benchmarks but programmed for Vivado HLS.
  \item \colorbox{lightGray}{benchmark/power\_measurement} is an ARM C application that should run on the second ARM core.
    It measures the power consumption during execution.
\end{enumerate}

\section{HW Synthesis}

\begin{itemize}
  \item Configure your FGPU HW parameters in \colorbox{lightGray}{RTL/FGPU\_definitions.vhd}
  \item Set the target clock frequency in \colorbox{lightGray}{scripts/tcl/implement\_FGPU.tcl}. 
    You may also modify the synthesis and implementation settings and the number of maximum number of threads used by Vivado in this file.
    FGPU can be synthesized at frequencies up to 250MHz.
  \item Run the implementation script \colorbox{lightGray}{scripts/tcl/implement\_FGPU.tcl}. 
    At the beginning, it calls \colorbox{lightGray}{scripts/tcl/create\_FGPU\_block\_design.tcl} 
    which creates a block diagram that connects FGPU to a clock generator and to the ARM processor\footnote{
      The block diagram can be found in \colorbox{lightGray}{HW/.srcs/sources\_1/bd/bd\_design}}.
    It uses the four available AXI4-HP interfaces to connect FGPU to the external SDRAM memory. 
    However, only the number of interfaces configured in \colorbox{lightGray}{RTL/FGPU\_definitions.vhd} will be used.
    All synthesis and implementation files can be found in 
    \colorbox{lightGray}{HW/synth} and \colorbox{lightGray}{HW/implement}, respectively.
    The generated bitstream and \emph{.hdf} files are stored in \colorbox{lightGray}{HW/outputs}.
\end{itemize}

\section{Write and Compile the Device Code}
\begin{itemize}
  \item Write your device code in OpenCL in a \emph{<benchmark>.cl} file in \colorbox{lightGray}{kernels} directory.
  \item Compile it with the \colorbox{lightGray}{compile.sh} or \colorbox{lightGray}{compile\_and\_log.sh} located in the same directory. 
    The compilation outputs can be found in \colorbox{lightGray}{kernels/outputs}.
    If hard floating-point support is desired, compile your code with the \colorbox{lightGray}{-hard-float} option.
\end{itemize}

\section{Write and Compile the Host Code}
\begin{itemize}
  \item Write your C/C++ application in a \colorbox{lightGray}{benchmark/<benchmark>/src/}.
    It should run on the first ARM core and it has to control FGPU.
    Please have a look at other benchmarks, e.g. \emph{xcorr}, and try to use the existing methods.
    Do not forget to copy the compiled device code from \colorbox{lightGray}{kernels/outputs/<benchmark>{.h,.c}} next to your C/C++ files.
    It will be downloaded to FGPU by the host application.
  \item You may compile your code by executing \colorbox{lightGray}{scripts/compile.sh}.
    Alternatively, you may execute \colorbox{lightGray}{make} from \colorbox{lightGray}{benchmark/<benchmark>/Release}.
    If you would like to use the GUI of Xilinx SDK, you should set the workspace to \colorbox{lightGray}{benchmark}.
\end{itemize}

\section{Download and Run}
  Unless you want to use the Xilinx SDK in GUI mode, you can run FGPU applications as follows:
\begin{itemize}
  \item Set the bitstream you want to download in \colorbox{lightGray}{benchmraks/program\_bitstream.tcl}
  \item From the \colorbox{lightGray}{benchmark} directory, open the SDK project you have compiled in the last step \\
    \centerline{\colorbox{lightGray}{./open\_sdk.sh <benchmark>}} \\
  This will open an xsct terminal.
  \item Make sure you have opened the terminal emulator and a connection to the board is established.
  \item Source the following scripts in the xsct terminal (Xilinx Software Command-Line Tool) \footnote{
    For further information about xsct, have a look at UG1208 (v2016.2)}\\
    \centerline{\colorbox{lightGray}{source program\_bitstream.tcl}} \\
    This will program the desired bitstream .
  \item Download the \emph{.elf} file of the corresponding SDK project by executing\\
    \centerline{\colorbox{lightGray}{source download\_elf.tcl}} \\
  Now you may observe the terminal output.
  \item If you would like to measure the power consumption during execution you have to:
    \begin{itemize}
      \item Activate the power measurement in the \emph{main.cpp} file of your application as follows \\
      \centerline{\colorbox{lightGray}{const unsigned sync\_power\_measurement = 1; }} \\
      This will enable the communication between the host application running on the first ARM core 
      and the power measurement running on the second one.
      Please pay attention that a single power measurement takes about 1.7ms to finish.
      Therefore, it does not make sense to compute small problems if a power measurement is being done.
    \item Recompile your host code.
    \item If you are using the GUI, you may download and execute the power measurement program before you do the same with the host application.
      Otherwise, it suffices to execute: \\
      \centerline{\colorbox{lightGray}{source measure\_power.tcl }} \\
    \end{itemize}
\end{itemize}

\end{document}
